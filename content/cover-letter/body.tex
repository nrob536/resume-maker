I am writing to express my interest in the \textbf{Senior Loss Modelling Analyst} position (\textit{Reference #: 29950}) at 
\textit{Natural Hazards Commission — Toka Tū Ake}.
I hold a PhD in Biological Sciences from the University of Auckland. I will bring over five years of experience in data analytics,
modelling, and risk-related projects across government and science agencies. I am passionate about delivering high-quality,
defensible analysis that informs resilience, readiness, and recovery.

In my current role as \textbf{Kaitātari Hoahoa (Design Analyst)} at \textit{Tatauranga Aotearoa — Stats NZ}, I lead the development 
of analytical tools, R packages, dashboards, and workflows that transform complex datasets into actionable insights for policy and
operational teams. I have managed and optimised large-scale data environments, including administering 200+ R Shiny applications, 
and have contributed to national statistical outputs such as census population modelling. 
My experience at \textit{National Institute of Atmospheric Research (NIWA)} as both a \textbf{Climate Database Technician} and 
\textbf{Molecular Biologist} further  strengthened my ability to design and refine models, ensure data quality, 
and communicate results to diverse stakeholders.

I bring proficiency in \textbf{Python, R, SQL,} and \textbf{geospatial analysis}, with a strong background in scenario modelling and data visualisation.
My academic and professional work has required synthesising complex model outputs into clear, actionable recommendations, aligning 
closely with the responsibilities of this role. I also have a proven track record of collaborating across agencies, having worked 
on multi-stakeholder projects in climate science, biosecurity, and environmental monitoring.

The values of hononga, māhirahira, and māia strongly resonate with both my personal and professional approach.
I believe meaningful outcomes start with connection, and I have consistently built strong, respectful relationships across
teams and agencies to ensure complex insights are accessible and actionable. My career has been driven by curiosity, 
from developing innovative ecological models during my PhD to adopting new tools like Python to expand my analytical capabilities.
I also bring confidence grounded in my ability to produce robust outcomes to inform decisions that impact communities across Aotearoa.

I would welcome the opportunity to discuss how my skills and experience can support your strategic objectives.